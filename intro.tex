% \chapter{Introduction}
\section{Motivation}

One of the most interesting aspects of chaos is the universality of many of the results.  One of the most famous examples is Fiebenbaum's proof of the quantitative universality of his scaling factors for all quadratic maps.  This result proves that measurements of $\alpha$ and $\delta$ from the bifurcation diagrams of vastly different physical systems should converge to the same universal values.  The beauty of that result is that despite the very different manifestations, one can show that in a way, most chaotic systems are just expressions of a more unified type of behavior, and still bear some of the quantifiable characteristics of that behavior.

In a similar vein, it can be shown that %reference here
the fractal dimension of a chaotic attractor is invariant under smooth coordinate changes %check the phrasing and conditions.  fixed-mass paper, first ref.
Thus, one can measure a quantity based on the measurements corresponding to a particular set of initial conditions, with some very specific embedding method, and still reach a result that characterizes the underlying dynamics in a way that is completely universal.

% Another universal quantity is the fractal dimension of a chaotic attractor, which is invariant under smooth coordinate transformations. %need a ref here.  Check the fixed-mass paper intro, which is where I am stealing this whole idea from.  While we're at it, reference feigenbaum.
% There are multiple definitions of dimension, and they give conflicting results.  However, a wide class of definitions can be unified with the multifractal spectrum.  Calculations of the resulting spectrum of dimensions can require vast amounts of data. %ref something here?

One of the potential problems with this idea is that there are multiple definitions of dimension, and these definitions are known to differ from one another.  This can be resolved with the introduction of the multifractal spectrum, which provides a unifying framework for a very broad class of fractal dimensions.  However, accurate calculations of the multifractal spectrum usually require prohibitive amounts of data, especially for high-dimensional attractors.  Our aim is to improve calculations of the multifractal spectrum for small data sets.